\section{Logistic Regression}
We can extend this learning to classification problems, where we have binary
labels $y$ that are either $0$ or $1$.
\subsection{The Logistic Function}
For logistic regression, our new hypothesis for estimating the class of a
sample $x$ is:
\begin{equation}
  h( x ) = g \left( \theta ^ { T } x \right) = \frac { 1 } { 1 + e ^ { - \theta ^ { T } x } }
\end{equation}
where $g(z)$ is the logistic or sigmoid function:
\begin{equation}
  g ( z ) = \frac { 1 } { 1 + e ^ { - z } }
\end{equation}
The sigmoid function is bounded between $0$ and $1$, and tends towards $1$ as
$z \rightarrow \infty$. It tends towards $0$ when $z \rightarrow -\infty$.
A useful property of the sigmoid function is the form of its derivative:
\begin{equation}
  \begin{aligned} g ^ { \prime } ( z ) \quad & = \quad \frac { d } { d z } \frac { 1 } { 1 + e ^ { - z } } \\
    & = \quad \frac { 1 } { \left( 1 + e ^ { - z } \right) ^ { 2 } } \left( e ^ { - z } \right) \\
    & = \quad \frac { 1 } { \left( 1 + e ^ { - z } \right) } \cdot \left( 1 - \frac { 1 } { \left( 1 + e ^ { - z } \right) } \right) \\
    & = \quad g ( z ) ( 1 - g ( z ) )
  \end{aligned}
\end{equation}
\subsection{Cost Function}
To fit $\theta$ for a set of training examples, we assume that:
\begin{equation}
  \begin{aligned}
    P ( y = 1 | x ; \theta ) & = h( x ) \\
    P ( y = 0 | x ; \theta ) & = 1 - h( x )
  \end{aligned}
\end{equation}
This can be written more compactly as:
\begin{equation}
  p ( y | x ; \theta ) = \left( h ( x ) \right) ^ { y } \left( 1 - h ( x ) \right) ^ { 1 - y }
\end{equation}
Assume $m$ training examples generated independently, we define the likelihood
function of the parameters as:
\begin{equation}
  \begin{aligned}
    L ( \theta ) \quad & = \quad p \left( \vec{y} | X ; \theta \right) \\
    & = \quad \prod _ { i = 1 } ^ { m } p \left( y ^ { ( i ) } | x ^ { ( i ) } ; \theta \right) \\
    & = \quad \prod _ { i = 1 } ^ { m } \left( h \left( x ^ { ( i ) } \right) \right) ^ { y ^ { ( i ) } } \left( 1 - h  \left( x ^ { ( i ) } \right) \right) ^ { 1 - y ^ { ( i ) } }
  \end{aligned}
\end{equation}
And taking the negative log likelihood to minimize:
\begin{equation}
  \begin{aligned}
    \ell ( \theta ) \quad & = \quad -\log L ( \theta ) \\
    & = \quad -\sum _ { i = 1 } ^ { m } y ^ { ( i ) } \log h \left( x ^ { ( i ) } \right) + \left( 1 - y ^ { ( i ) } \right) \log \left( 1 - h \left( x ^ { ( i ) } \right) \right) \\
    & = \quad -\sum_{i=1}^{m} y^{(i)} \theta^T x^{(i)} - \log \left( 1 + e^{\theta^T x^{(i)}} \right)
  \end{aligned}
\end{equation}
This is known as the binary cross-entropy loss function.
\subsection{Gradient Descent}
Let’s start by working with just one training example (x,y), and take
derivatives to derive the stochastic gradient ascent rule:
\begin{equation}
  \begin{aligned}
    \frac { \partial } { \partial \theta _ { j } } \ell ( \theta ) \quad & = \quad - \left( y \frac { 1 } { g \left( \theta ^ { T } x \right) } - ( 1 - y ) \frac { 1 } { 1 - g \left( \theta ^ { T } x \right) } \right) \frac { \partial } { \partial \theta _ { j } } g \left( \theta ^ { T } x \right) \\
    & = \quad - \left( y \frac { 1 } { g \left( \theta ^ { T } x \right) } - ( 1 - y ) \frac { 1 } { 1 - g \left( \theta ^ { T } x \right) } \right) g \left( \theta ^ { T } x \right) \left( 1 - g \left( \theta ^ { T } x \right) \right) \frac { \partial } { \partial \theta _ { j } } \theta ^ { T } x \\
    & = \quad - \left( y \left( 1 - g \left( \theta ^ { T } x \right) \right) - ( 1 - y ) g \left( \theta ^ { T } x \right) \right) x _ { j } \\
    & = \quad - \left( y - h ( x ) \right) x _ { j }
  \end{aligned}
\end{equation}
This therefore gives us the stochastic gradient ascent rule:
\begin{equation}
  \theta _ { j } : = \theta _ { j } + \alpha \left( y ^ { ( i ) } - h \left( x ^ { ( i ) } \right) \right) x _ { j } ^ { ( i ) }
\end{equation}
We must use gradient descent for logistic regression since there is no closed
form solution for this problem.
\subsection{Newton-Raphson Algorithm}
The Hessian matrix for logistic regression is:
\begin{equation}
  \frac{\partial^2 \ell(\theta)}{\partial \theta \partial \theta} = -\sum_{i=1}^m x^{(i)} x^{(i) T} \cdot h\left(x^{(i)}\right) \cdot \left(1-h\left(x^{(i)}\right)\right)
\end{equation}
Hence the second order update is:
\begin{equation}
  \theta := \theta - \left( \frac{\partial^2 \ell(\theta)}{\partial \theta \partial \theta} \right)^{-1} \frac{\partial \ell(\theta)}{\partial \theta}
\end{equation}
