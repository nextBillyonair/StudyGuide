%------------------------------------------------------------------------------
%   PACKAGES AND OTHER DOCUMENT CONFIGURATIONS
%------------------------------------------------------------------------------

\documentclass[twoside,twocolumn]{article}

%\usepackage[sc]{mathpazo} % Use the Palatino font
%\usepackage[T1]{fontenc} % Use 8-bit encoding that has 256 glyphs
%\linespread{1.05} % Line spacing - Palatino needs more space between lines
%\usepackage{microtype} % Slightly tweak font spacing for aesthetics

\usepackage[english]{babel} % Language hyphenation and typographical rules

% Document margins
\usepackage[hmarginratio=1:1,top=32mm,left=20mm,right=20mm,columnsep=20pt]{geometry}
% Custom captions under/above floats in tables or figures
\usepackage[hang, small,labelfont=bf,up,textfont=it,up]{caption}
\usepackage{booktabs} % Horizontal rules in tables

\usepackage{enumitem} % Customized lists
\setlist[itemize]{noitemsep} % Make itemize lists more compact
\usepackage{textcomp}

% Allows abstract customization
\usepackage{abstract}
% Set the "Abstract" text to bold
\renewcommand{\abstractnamefont}{\normalfont\bfseries}
% Set the abstract itself to small italic text
\renewcommand{\abstracttextfont}{\normalfont\small\itshape}

\usepackage{fancyhdr} % Headers and footers
\pagestyle{fancy} % All pages have headers and footers
\fancyhead{} % Blank out the default header
\fancyfoot{} % Blank out the default footer
\fancyhead[C]{\thetitle}
\fancyfoot[RO,LE]{\thepage} % Custom footer text

\usepackage{titling} % Customizing the title section

\usepackage{hyperref} % For hyperlinks in the PDF
\usepackage{amsmath}
\usepackage{amssymb}

\usepackage{tikz}
\usetikzlibrary{bayesnet, arrows, positioning, fit, arrows.meta, shapes}

\usepackage{color}
\usepackage{caption}
\usepackage{subcaption}

\usepackage{graphicx}


\captionsetup[figure]{labelfont={bf},textfont=normalfont}

%------------------------------------------------------------------------------
%   TITLE SECTION
%------------------------------------------------------------------------------
\newlength\mystoreparindent
\newenvironment{myparindent}[1]{%
  \setlength{\mystoreparindent}{\the\parindent}
  \setlength{\parindent}{#1}
  }{%
  \setlength{\parindent}{\mystoreparindent}
}


%% independence symbol and expectation operator %%
\newcommand\independent{\protect\mathpalette{\protect\independenT}{\perp}}
\def\independenT#1#2{\mathrel{\rlap{$#1#2$}\mkern2mu{#1#2}}}

\DeclareMathOperator*{\E}{\mathbb{E}}
\DeclareMathOperator*{\Et}{\mathbb{E}_t}
\DeclareMathOperator*{\argmax}{arg\,max}
\DeclareMathOperator{\circlearrow}{\hbox{$\circ$}\kern-1.5pt\hbox{$\rightarrow$}}
\DeclareMathOperator{\circlecircle}{\hbox{$\circ$}\kern-1.2pt\hbox{$--$}\kern-1.5pt\hbox{$\circ$}}

\newcommand\indep{\protect\mathpalette{\protect\independenT}{\perp}}
\def\independenT#1#2{\mathrel{\rlap{$#1#2$}\mkern2mu{#1#2}}}

\setlength{\droptitle}{-4\baselineskip} % Move the title up

\pretitle{\begin{center}\Huge\bfseries} % Article title formatting
\posttitle{\end{center}} % Article title closing formatting

\title{Top Equations in Deep Learning}
% \author{%
%   \textsc{William Watson} \\[1ex]
%   \normalsize Johns Hopkins University \\
%   \normalsize \href{mailto:billwatson@jhu.edu}{billwatson@jhu.edu}
% }

\date{\today} % Leave empty to omit a date

% \renewcommand{\maketitlehookd}{%

% }

\newcommand{\specialcell}[2][c]{%
  \begin{tabular}[#1]{@{}l@{}}#2\end{tabular}}

%------------------------------------------------------------------------------

\begin{document}

% Print the title
\maketitle

%------------------------------------------------------------------------------
%   ARTICLE CONTENTS
%------------------------------------------------------------------------------
% \tableofcontents
%------------------------------------------------
\onecolumn
%------------------------------------------------

\begin{table}
  \centering
  \resizebox{\columnwidth}{!}{
  \begin{tabular}{p{0.2cm} p{6cm} p{12cm} p{6cm} p{4cm}}
    \toprule
    {} & Equation & {} & Use & Citation \\ \\
    \midrule
    1  &  Stochastic Gradient Descent
      & $\displaystyle \theta_{i+1} \leftarrow \theta_{i} - \alpha \sum_{x \in B} \frac{\partial L(x)}{\partial \theta}$      & Optimization      &
    \\ \\ \\
    2  &  Backpropagation
      & $\displaystyle \frac{\partial L}{\partial \theta} = \frac{\partial L}{\partial z}\frac{\partial z}{\partial \theta} $      & Computing Gradients      &
    \\ \\ \\
    3  &  Linear Layer
      & $\displaystyle y = Wx + b$      & Base Layer in Deep Learning      &
    \\ \\ \\
    4  &  Convolutions
      &   $\displaystyle (f \ast g)[n] = \sum_{m=0}^{K} f[n - m] g[m] $    & Image Networks      &
    \\ \\ \\
    5  &  ReLU
      & $\displaystyle \operatorname{ReLU}\left( x \right) = \max(0, x)$      & Activation Function      &
    \\ \\ \\
    6  &  Recurrent Cells
      & $\displaystyle h_{t+1} = \tanh \left( W_{ih} x_t + b_{ih} + W_{hh}h_t + b_{hh} \right)$      & Sequence Modeling      &
    \\ \\ \\
    7  &  Self Attention
      & $\displaystyle \operatorname{Attention} \left( Q, K, V \right) = \operatorname{softmax} \left( \frac{QK^T}{\sqrt{d_k}} \right) V$      & Transformers      &
    \\ \\ \\
    8  &  Generative Adversarial Netowrks
      & $\displaystyle \min_G \max_D \mathop{\mathbb{E}}_{x \sim p_{data}(x)}\left[\log D(x)\right] + \mathop{\mathbb{E}}_{z \sim p_{z}(z)}\left[\log \left(1 - D(G(z))\right) \right]  $      & Generative Modeling      &
    \\ \\ \\
    9  &  REINFORCE
      & $\displaystyle \triangle \theta = \alpha r \frac{\partial \log p \left( a \mid \pi^{\theta}(s) \right)}{\partial \theta}$      & Reinforcement Learning       &
    \\ \\ \\
    10 &  (Variational) Auto-Encoder
      &       & Liklihood Modeling       &
    \\ \\ \\
    11 &  Categorical Cross Entropy
      & $\displaystyle L\left(x, y\right) = -x_y + \log \left(\sum_j \exp \left(x_j\right)\right)$      & Multiclass Classification      &
    \\ \\ \\
    12 &  Bellman Equation
      &
      \specialcell{$\displaystyle V(s_t) = \max_{a_t} \left( R(s_t, a_t) + \gamma \sum_{s_{t+1}} p(s_t, a_t, s_{t+1}) V(s_{t+1}) \right)$\\$\displaystyle Q(s_t, a_t) = Q(s_t, a_t) + \alpha \cdot \left( R(s_t, a_t) + \gamma \cdot \max_{a} Q \left(s_{t+1}, a \right) - Q \left( s_t, a_t \right) \right)$}
      &  Reinforcement Learning     &
    \\ \\ \\
    13 &  Reparametrization Trick &
    $\displaystyle \mathop{\mathbb{E}}_{z \sim p(z|\theta)} \left[ f(z) \right] = \mathop{\mathbb{E}}_{\epsilon \sim p(\epsilon)} \left[ f(g(\epsilon, \theta)) \right]$ & Differential Sampling
    % $\displaystyle \tilde{x} \sim \mu + \sigma \mathcal{N}(0, 1)$  &
           &
    \\ \\ \\
    14 &  $\ell_p$ Norm
      & $\displaystyle \| x  \|_p = \left( \sum_{i=1}^n |x_i|^p \right)^{1/p} \qquad \| A \|_F \sqrt{\sum_{i=1}^m \sum_{j=1}^n |a_{ij}|^2 }$
            & Distance, Regression Loss      &
    \\ \\ \\
    15 &  Batch Normalization
      & $\displaystyle y = \frac{x - \mathop{\mathbb{E}} \left[ x \right]}{\sqrt{\operatorname{Var}\left[ x \right] + \epsilon}} \ast \gamma + \beta$      & Regularization      &
    \\ \\ \\
    16 &  Mutual Information
      &       &       &
    \\
    \bottomrule
  \end{tabular}
  }
  \caption{}
  \label{}

\end{table}


% References
% \bibliographystyle{abbrv}
% \bibliography{study_guide}

%------------------------------------------------

\end{document}
